\begin{resumo}

No campo da previsão de séries temporais os métodos mais difundidos baseiam-se em predição por ponto. Esse tipo de previsão, no entanto, tem um sério inconveniente: ele não quantifica as incertezas inerentes aos processos naturais e sociais nem outras incertezas decorrentes da captura e processamento dos dados. Por isso nos últimos anos os métodos de previsão intervalar e probabilística têm ganhado a atenção dos pesquisadores, particularmente nas ciências climáticas e na econometria. Mas outro inconveniente vem do fato de grande parte dos métodos de previsão probabilística serem métodos de caixa preta e demandarem simulações estocásticas ou \textit{ensembles} de métodos preditivos que são computacionalmente despendiosos.

Por outro lado, o volume (número de registros) e a dimensionalidade (número de variáveis) dos dados vêm alcançando magnitudes cada vez maiores, graças ao barateamento dos dispositivos computacionais de captura e armazenamento de dados, um fenômeno conhecido como \textit{Big Data}. Tais fatores impactam diretamente no custo de treinamento e atualização dos modelos e, para séries temporais com essas características, a escalabilidade tornou-se um fator decisivo na escolha dos métodos preditivos.

Nesse contexto emergem os métodos de Séries Temporais Nebulosas, que vêm em crescente expansão nos últimos anos dado os seus resultados acurados, a facilidade de implementação dos métodos, o seu baixo custo computacional e a interpretabilidade de seus modelos. Os métodos de Séries Temporais Nebulosas têm sido utilizados em áreas como previsão de demanda energética, indicadores e ativos de mercado, turismo entre outras. Mas há lacunas na literatura de tais métodos referentes a escalabilidade para grandes volumes de dados e  previsão probabilística e por intervalos.

A presente tese propõe novos métodos escaláveis de Séries Temporais Nebulosas e investiga a aplicação desses modelos na previsão por ponto, intervalar e probabilística, para uma ou mais variáveis e para mais de um passo à frente. Os parâmetros e hiperparâmetros dos métodos são discutidos e são apresentadas alternativas de ajuste fino dos modelos. Os métodos propostos são então comparados com as principais técnicas de Séries Temporais Nebulosas e outros modelos estatísticos utilizando dados ambientais e do mercado de ações. Os modelos propostos apresentaram resultados promissores tanto nas previsões por ponto quanto nas previsões por intervalo e probabilísticas e com baixo custo computacional, tornando-os úteis para um vasta gama de aplicações.
        
Palavras-chave: Séries Temporais Nebulosas, Previsão Probabilística, Escalabilidade, Previsão por Intervalo.

\end{resumo}

%-----------------------------------
% Abstract (in English)
\begin{resumo}[Abstract]
    \begin{otherlanguage*}{english}

In the field of time series forecasting, the most known methods are based on point forecasting. However, this kind of forecasting has a serious drawback: it does not quantify the uncertainties inherent to natural and social processes neither other uncertainties caused by the data gathering and processing. Because this in last years the interval and probabilistic forecasting methods have been gaining more attention of researches, specially on environmental and economical sciences. But these techniques also have their own issues due to the methods being black-boxes and requiring stochastic simulations and ensembles of multiple forecasting methods which are computationally expensive.

On the other hand, the data volume (number of instances) and dimensionality (number of variables) have reached magnitudes even greater, due to the commoditizing of the capturing and storing computational devices, in a phenomenon known as Big Data. Such factors impact directly on the model's training and updating costs, and for time series with Big Data characteristics, the scalability became a decisive factor in the choosing of predictive methods.

In this context the Fuzzy Time Series (FTS) methods emerge, which have been growing in recent years due to their accurate results, easiness of implementation, low computational cost and model explainability. The Fuzzy Time Series methods have been applied to forecast electric load, market assets, economical indicators, tourism demand etc. But there is a lack on FTS literature regarding interval and probabilistic forecasting.

This thesis proposes new scalable Fuzzy Time Series methods and discusses its application to  point, interval and probabilistic forecasting of mono and multivariate time series, for one to many steps ahead. The parameters and hyper-parameters are discussed and fine tunning alternatives are presented. Finally the proposed methods are compared with the main Fuzzy Time Series techniques and other literature approaches using environmental and stock market data. The proposed methods obtained promising results on point, interval and probabilistic forecasting and presented low computational cost, making it useful for a wide range of applications.

\textit{Keywords: Fuzzy Time Series, Probabilistic Forecasting, Interval Forecasting, Scalable Models.}
    \end{otherlanguage*}
\end{resumo}